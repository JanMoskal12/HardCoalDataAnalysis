% Options for packages loaded elsewhere
\PassOptionsToPackage{unicode}{hyperref}
\PassOptionsToPackage{hyphens}{url}
%
\documentclass[
]{article}
\usepackage{amsmath,amssymb}
\usepackage{iftex}
\ifPDFTeX
  \usepackage[T1]{fontenc}
  \usepackage[utf8]{inputenc}
  \usepackage{textcomp} % provide euro and other symbols
\else % if luatex or xetex
  \usepackage{unicode-math} % this also loads fontspec
  \defaultfontfeatures{Scale=MatchLowercase}
  \defaultfontfeatures[\rmfamily]{Ligatures=TeX,Scale=1}
\fi
\usepackage{lmodern}
\ifPDFTeX\else
  % xetex/luatex font selection
\fi
% Use upquote if available, for straight quotes in verbatim environments
\IfFileExists{upquote.sty}{\usepackage{upquote}}{}
\IfFileExists{microtype.sty}{% use microtype if available
  \usepackage[]{microtype}
  \UseMicrotypeSet[protrusion]{basicmath} % disable protrusion for tt fonts
}{}
\makeatletter
\@ifundefined{KOMAClassName}{% if non-KOMA class
  \IfFileExists{parskip.sty}{%
    \usepackage{parskip}
  }{% else
    \setlength{\parindent}{0pt}
    \setlength{\parskip}{6pt plus 2pt minus 1pt}}
}{% if KOMA class
  \KOMAoptions{parskip=half}}
\makeatother
\usepackage{xcolor}
\usepackage[margin=1in]{geometry}
\usepackage{color}
\usepackage{fancyvrb}
\newcommand{\VerbBar}{|}
\newcommand{\VERB}{\Verb[commandchars=\\\{\}]}
\DefineVerbatimEnvironment{Highlighting}{Verbatim}{commandchars=\\\{\}}
% Add ',fontsize=\small' for more characters per line
\usepackage{framed}
\definecolor{shadecolor}{RGB}{248,248,248}
\newenvironment{Shaded}{\begin{snugshade}}{\end{snugshade}}
\newcommand{\AlertTok}[1]{\textcolor[rgb]{0.94,0.16,0.16}{#1}}
\newcommand{\AnnotationTok}[1]{\textcolor[rgb]{0.56,0.35,0.01}{\textbf{\textit{#1}}}}
\newcommand{\AttributeTok}[1]{\textcolor[rgb]{0.13,0.29,0.53}{#1}}
\newcommand{\BaseNTok}[1]{\textcolor[rgb]{0.00,0.00,0.81}{#1}}
\newcommand{\BuiltInTok}[1]{#1}
\newcommand{\CharTok}[1]{\textcolor[rgb]{0.31,0.60,0.02}{#1}}
\newcommand{\CommentTok}[1]{\textcolor[rgb]{0.56,0.35,0.01}{\textit{#1}}}
\newcommand{\CommentVarTok}[1]{\textcolor[rgb]{0.56,0.35,0.01}{\textbf{\textit{#1}}}}
\newcommand{\ConstantTok}[1]{\textcolor[rgb]{0.56,0.35,0.01}{#1}}
\newcommand{\ControlFlowTok}[1]{\textcolor[rgb]{0.13,0.29,0.53}{\textbf{#1}}}
\newcommand{\DataTypeTok}[1]{\textcolor[rgb]{0.13,0.29,0.53}{#1}}
\newcommand{\DecValTok}[1]{\textcolor[rgb]{0.00,0.00,0.81}{#1}}
\newcommand{\DocumentationTok}[1]{\textcolor[rgb]{0.56,0.35,0.01}{\textbf{\textit{#1}}}}
\newcommand{\ErrorTok}[1]{\textcolor[rgb]{0.64,0.00,0.00}{\textbf{#1}}}
\newcommand{\ExtensionTok}[1]{#1}
\newcommand{\FloatTok}[1]{\textcolor[rgb]{0.00,0.00,0.81}{#1}}
\newcommand{\FunctionTok}[1]{\textcolor[rgb]{0.13,0.29,0.53}{\textbf{#1}}}
\newcommand{\ImportTok}[1]{#1}
\newcommand{\InformationTok}[1]{\textcolor[rgb]{0.56,0.35,0.01}{\textbf{\textit{#1}}}}
\newcommand{\KeywordTok}[1]{\textcolor[rgb]{0.13,0.29,0.53}{\textbf{#1}}}
\newcommand{\NormalTok}[1]{#1}
\newcommand{\OperatorTok}[1]{\textcolor[rgb]{0.81,0.36,0.00}{\textbf{#1}}}
\newcommand{\OtherTok}[1]{\textcolor[rgb]{0.56,0.35,0.01}{#1}}
\newcommand{\PreprocessorTok}[1]{\textcolor[rgb]{0.56,0.35,0.01}{\textit{#1}}}
\newcommand{\RegionMarkerTok}[1]{#1}
\newcommand{\SpecialCharTok}[1]{\textcolor[rgb]{0.81,0.36,0.00}{\textbf{#1}}}
\newcommand{\SpecialStringTok}[1]{\textcolor[rgb]{0.31,0.60,0.02}{#1}}
\newcommand{\StringTok}[1]{\textcolor[rgb]{0.31,0.60,0.02}{#1}}
\newcommand{\VariableTok}[1]{\textcolor[rgb]{0.00,0.00,0.00}{#1}}
\newcommand{\VerbatimStringTok}[1]{\textcolor[rgb]{0.31,0.60,0.02}{#1}}
\newcommand{\WarningTok}[1]{\textcolor[rgb]{0.56,0.35,0.01}{\textbf{\textit{#1}}}}
\usepackage{graphicx}
\makeatletter
\def\maxwidth{\ifdim\Gin@nat@width>\linewidth\linewidth\else\Gin@nat@width\fi}
\def\maxheight{\ifdim\Gin@nat@height>\textheight\textheight\else\Gin@nat@height\fi}
\makeatother
% Scale images if necessary, so that they will not overflow the page
% margins by default, and it is still possible to overwrite the defaults
% using explicit options in \includegraphics[width, height, ...]{}
\setkeys{Gin}{width=\maxwidth,height=\maxheight,keepaspectratio}
% Set default figure placement to htbp
\makeatletter
\def\fps@figure{htbp}
\makeatother
\setlength{\emergencystretch}{3em} % prevent overfull lines
\providecommand{\tightlist}{%
  \setlength{\itemsep}{0pt}\setlength{\parskip}{0pt}}
\setcounter{secnumdepth}{-\maxdimen} % remove section numbering
\ifLuaTeX
  \usepackage{selnolig}  % disable illegal ligatures
\fi
\usepackage{bookmark}
\IfFileExists{xurl.sty}{\usepackage{xurl}}{} % add URL line breaks if available
\urlstyle{same}
\hypersetup{
  pdftitle={Projekt z Szeregów Czasowych},
  pdfauthor={Jan Moskal},
  hidelinks,
  pdfcreator={LaTeX via pandoc}}

\title{Projekt z Szeregów Czasowych}
\author{Jan Moskal}
\date{2025-01-18}

\begin{document}
\maketitle

\subsubsection{Wstęp}\label{wstux119p}

Dane, które będziemy analizować, pochodzą ze strony Głównego Urzędu
Statystycznego (\url{https://bdl.stat.gov.pl/bdl/dane/podgrup/temat})
znajdują się w grupie ``Przeciętne ceny detaliczne towarów i usług
konsumpcyjnych'', podgrupie ``Ceny detaliczne wybranych towarów i usług
konsumpcyjnych (dane miesięczne)'' i dotyczą cen węgla kamiennego za
toną. Dane o przeciętnych cenach obejmują notowania co miesiąc dla całej
Polski. Projekt ma na celu analizę tego szeregu czasowego, aby zrozumieć
zmiany cen węgla kamiennego w Polsce w latach 2006-2019 i stworzyć
prognozy na przyszłość.

\subsubsection{Wczytywanie danych}\label{wczytywanie-danych}

\begin{Shaded}
\begin{Highlighting}[]
\NormalTok{dane }\OtherTok{\textless{}{-}} \FunctionTok{read\_excel}\NormalTok{(}\StringTok{"wegiel\_kamienny\_szereg.xlsx"}\NormalTok{, }\AttributeTok{range =} \StringTok{"TABLICA!C4:FN6"}\NormalTok{)}

\NormalTok{dane }\OtherTok{\textless{}{-}} \FunctionTok{as.vector}\NormalTok{(dane[}\DecValTok{2}\NormalTok{, ])}
\NormalTok{dane }\OtherTok{\textless{}{-}} \FunctionTok{as.numeric}\NormalTok{(}\FunctionTok{unlist}\NormalTok{(dane))}
\end{Highlighting}
\end{Shaded}

Zamienimay wektor w macierz, aby ustawić dobrą kolejność danych (narazie
mamy dane wypisane, w ten sposób, że jeden miesiąc dla czternastu lat i
dopiero następny miesiąc, a chcemy żeby było chronologicznie)

\begin{Shaded}
\begin{Highlighting}[]
\NormalTok{macierz }\OtherTok{\textless{}{-}} \FunctionTok{matrix}\NormalTok{(dane, }\AttributeTok{ncol =} \DecValTok{14}\NormalTok{, }\AttributeTok{byrow =} \ConstantTok{TRUE}\NormalTok{)}
\end{Highlighting}
\end{Shaded}

Przekształcenie macierzy w wektor czytany kolumnowo (od góry do dołu). W
ten sposób otrzymujemy dane w odpowiedniej kolejności.

\begin{Shaded}
\begin{Highlighting}[]
\NormalTok{dane }\OtherTok{\textless{}{-}} \FunctionTok{as.vector}\NormalTok{(macierz)}
\NormalTok{t }\OtherTok{\textless{}{-}} \DecValTok{1}\SpecialCharTok{:}\FunctionTok{length}\NormalTok{(dane)}
\end{Highlighting}
\end{Shaded}

\subsubsection{Wstępna analiza
szeregu}\label{wstux119pna-analiza-szeregu}

Wykres liniowy dla naszych danych w czasie t.

\begin{Shaded}
\begin{Highlighting}[]
\FunctionTok{plot}\NormalTok{(}\AttributeTok{y=}\NormalTok{dane,}\AttributeTok{x=}\NormalTok{t,}\AttributeTok{col=}\DecValTok{4}\NormalTok{, }\AttributeTok{main =}\StringTok{"Cena kukurydzy"}\NormalTok{, }\AttributeTok{type =}\StringTok{"l"}\NormalTok{, }\AttributeTok{xlab =} \StringTok{"Numer tygodnia"}\NormalTok{ )}
\end{Highlighting}
\end{Shaded}

\includegraphics{HardCoalDataAnalysis_files/figure-latex/unnamed-chunk-6-1.pdf}

Jak widzimy z wykresu, cena dość szybko wzrosła do cen powyżej 700 zł.
Widzimy również, że ogólny trend jest rosnący.

Robimy wykresy typu boxplot oraz histogram, żeby zobaczyć rozkład
danych.

\begin{Shaded}
\begin{Highlighting}[]
\FunctionTok{par}\NormalTok{(}\AttributeTok{mfrow =} \FunctionTok{c}\NormalTok{(}\DecValTok{1}\NormalTok{, }\DecValTok{2}\NormalTok{)) }
\FunctionTok{hist}\NormalTok{(dane, }\AttributeTok{breaks =} \DecValTok{15}\NormalTok{, }\AttributeTok{col =} \DecValTok{4}\NormalTok{, }\AttributeTok{main =} \StringTok{"Histogram cen węgla kamiennego"}\NormalTok{,}
     \AttributeTok{xlab =} \StringTok{"Cena w Złoty"}\NormalTok{, }\AttributeTok{ylab =} \StringTok{"Liczebność"}\NormalTok{, }\AttributeTok{prob =} \ConstantTok{FALSE}\NormalTok{)}
\FunctionTok{boxplot}\NormalTok{(dane, }\AttributeTok{col =} \StringTok{"lightgrey"}\NormalTok{, }\AttributeTok{main =} \StringTok{"Wykres ramka{-}wąsy"}\NormalTok{, }\AttributeTok{horizontal =} \ConstantTok{TRUE}\NormalTok{)}
\end{Highlighting}
\end{Shaded}

\includegraphics{HardCoalDataAnalysis_files/figure-latex/unnamed-chunk-7-1.pdf}

\begin{Shaded}
\begin{Highlighting}[]
\FunctionTok{par}\NormalTok{(}\AttributeTok{mfrow =} \FunctionTok{c}\NormalTok{(}\DecValTok{1}\NormalTok{, }\DecValTok{1}\NormalTok{))  }
\end{Highlighting}
\end{Shaded}

Możemy zauważyć, że rozkład jest lewostronnie asymetryczny, bierzę się
to z tego co już zauważyliśmy z wykresu liniowego czyli, że ceny od 700
zł za tonę zaczęły się już po 2 latach od pierwszej obserwacji z szeregu
a pozostałe 12 lat oscylowało co do wartości od 700 do 900 zł za tonę. Z
wykresu pudełkowego możemu zauważyć nawet dokładniej, że kwartyl
pierwszy wynosi około 700 a kwartyl trzeci około 810 co w przełożeniu na
nasz problem oznacza, że połowa obserwacji, czyli z 7 lat znajduje się
na tym małym przedziale.

Podstawowe statystyki

\begin{Shaded}
\begin{Highlighting}[]
\FunctionTok{summary}\NormalTok{(dane)}
\end{Highlighting}
\end{Shaded}

\begin{verbatim}
##    Min. 1st Qu.  Median    Mean 3rd Qu.    Max. 
##   477.1   702.2   788.7   744.1   814.5   897.3
\end{verbatim}

Szukamy najlepszego wielomianu opisującego nasz szereg

\begin{Shaded}
\begin{Highlighting}[]
\NormalTok{Akaike }\OtherTok{\textless{}{-}} \FunctionTok{c}\NormalTok{()}
\ControlFlowTok{for}\NormalTok{(i }\ControlFlowTok{in} \DecValTok{1}\SpecialCharTok{:}\DecValTok{30}\NormalTok{)\{}
\NormalTok{  Akaike }\OtherTok{\textless{}{-}} \FunctionTok{cbind}\NormalTok{(Akaike, }\FunctionTok{dopasowanie\_wielomianu}\NormalTok{(dane,i))}
\NormalTok{\}}
\NormalTok{i}\OtherTok{=}\DecValTok{1}\SpecialCharTok{:}\DecValTok{30}
\FunctionTok{plot}\NormalTok{(i, Akaike,}\AttributeTok{type=}\StringTok{"p"}\NormalTok{, }\AttributeTok{pch=}\DecValTok{19}\NormalTok{, }\AttributeTok{main =}\StringTok{"Kryterium AIC dla wielomianu stopnia i"}\NormalTok{, }\AttributeTok{xlab =} \StringTok{"Stopień wielomianu"}\NormalTok{ )}
\end{Highlighting}
\end{Shaded}

\includegraphics{HardCoalDataAnalysis_files/figure-latex/unnamed-chunk-9-1.pdf}

Z kryterium osuwiska wybieramy wielomian stopnia 5.

\begin{Shaded}
\begin{Highlighting}[]
  \FunctionTok{par}\NormalTok{(}\AttributeTok{mfrow =} \FunctionTok{c}\NormalTok{(}\DecValTok{1}\NormalTok{, }\DecValTok{3}\NormalTok{))}
\NormalTok{  model\_st\_5 }\OtherTok{\textless{}{-}} \FunctionTok{lm}\NormalTok{(dane }\SpecialCharTok{\textasciitilde{}}\NormalTok{ ., }\AttributeTok{data=}\NormalTok{ramka)}
  \FunctionTok{plot}\NormalTok{(t, dane, }\AttributeTok{type =} \StringTok{"l"}\NormalTok{,}
       \AttributeTok{main =} \FunctionTok{paste}\NormalTok{(}\StringTok{"Dopasowanie wiel. st.:"}\NormalTok{, }\DecValTok{5}\NormalTok{),}
       \AttributeTok{ylab =} \StringTok{"Złoty"}\NormalTok{, }\AttributeTok{xlab =} \StringTok{"Cena"}\NormalTok{)}
  \FunctionTok{lines}\NormalTok{(t, model\_st\_5}\SpecialCharTok{$}\NormalTok{fitted.values, }\AttributeTok{col =} \DecValTok{2}\NormalTok{, }\AttributeTok{lwd =} \FloatTok{1.5}\NormalTok{)}

  \FunctionTok{plot}\NormalTok{(t, model\_st\_5}\SpecialCharTok{$}\NormalTok{residuals, }\AttributeTok{main =} \StringTok{"Reszty"}\NormalTok{, }\AttributeTok{type =} \StringTok{"l"}\NormalTok{)}
  \FunctionTok{abline}\NormalTok{(}\AttributeTok{h =} \DecValTok{0}\NormalTok{, }\AttributeTok{col =} \DecValTok{2}\NormalTok{, }\AttributeTok{lwd =} \DecValTok{2}\NormalTok{)}

  \FunctionTok{plot}\NormalTok{(}\FunctionTok{ecdf}\NormalTok{(model\_st\_5}\SpecialCharTok{$}\NormalTok{residuals), }\AttributeTok{main =} \StringTok{"Dystrybuanta"}\NormalTok{)}
\NormalTok{  x }\OtherTok{\textless{}{-}} \FunctionTok{seq}\NormalTok{(}\AttributeTok{from =} \FunctionTok{min}\NormalTok{(model\_st\_5}\SpecialCharTok{$}\NormalTok{residuals), }\AttributeTok{to =} \FunctionTok{max}\NormalTok{(model\_st\_5}\SpecialCharTok{$}\NormalTok{residuals), }\AttributeTok{length.out =} \DecValTok{500}\NormalTok{)}
  \FunctionTok{lines}\NormalTok{(x, }\FunctionTok{pnorm}\NormalTok{(x, }\AttributeTok{mean =} \DecValTok{0}\NormalTok{, }\AttributeTok{sd =} \FunctionTok{sd}\NormalTok{(model\_st\_5}\SpecialCharTok{$}\NormalTok{residuals)), }\AttributeTok{col =} \DecValTok{4}\NormalTok{, }\AttributeTok{lwd =} \DecValTok{2}\NormalTok{)}
\end{Highlighting}
\end{Shaded}

\includegraphics{HardCoalDataAnalysis_files/figure-latex/unnamed-chunk-11-1.pdf}

\begin{Shaded}
\begin{Highlighting}[]
  \FunctionTok{par}\NormalTok{(}\AttributeTok{mfrow =} \FunctionTok{c}\NormalTok{(}\DecValTok{1}\NormalTok{, }\DecValTok{1}\NormalTok{))}
\end{Highlighting}
\end{Shaded}

\subsubsection{Metoda wykładniczych wag ruchomej
średniej}\label{metoda-wykux142adniczych-wag-ruchomej-ux15bredniej}

\begin{Shaded}
\begin{Highlighting}[]
\FunctionTok{plot}\NormalTok{(dane, }\AttributeTok{main =} \StringTok{"Ceny węgla kamiennego"}\NormalTok{, }\AttributeTok{lwd =} \DecValTok{2}\NormalTok{, }\AttributeTok{type =} \StringTok{"l"}\NormalTok{, }\AttributeTok{xlab =} \StringTok{"Czas"}\NormalTok{, }\AttributeTok{ylab =} \StringTok{"zł"}\NormalTok{)}
\NormalTok{wykładnicza(dane, }\FloatTok{0.9}\NormalTok{, }\StringTok{"red"}\NormalTok{)}
\NormalTok{wykładnicza(dane, }\FloatTok{0.7}\NormalTok{, }\StringTok{"blue"}\NormalTok{)}
\NormalTok{wykładnicza(dane, }\FloatTok{0.5}\NormalTok{, }\StringTok{"green"}\NormalTok{)}
\NormalTok{wykładnicza(dane, }\FloatTok{0.3}\NormalTok{, }\StringTok{"purple"}\NormalTok{)}
\end{Highlighting}
\end{Shaded}

\includegraphics{HardCoalDataAnalysis_files/figure-latex/unnamed-chunk-12-1.pdf}

W celu elminacji losowych fluktuacji w szergeu czasowym użyliśmy metody
wykładniczych wag ruchomej średniej. Widać, że wraz ze wzrostem
współczynnika wag funkcja przybliżająca mniej naśladuje zaburzenia
zewnętrzne w modelu i na odwrót.

\subsubsection{Analiza trendów
fazowych}\label{analiza-trenduxf3w-fazowych}

\begin{Shaded}
\begin{Highlighting}[]
\NormalTok{w }\OtherTok{\textless{}{-}} \FunctionTok{ts}\NormalTok{(dane, }\AttributeTok{frequency =} \DecValTok{12}\NormalTok{)}
\FunctionTok{plot}\NormalTok{(w, }\AttributeTok{type =} \StringTok{"l"}\NormalTok{)}
\end{Highlighting}
\end{Shaded}

\includegraphics{HardCoalDataAnalysis_files/figure-latex/unnamed-chunk-13-1.pdf}

\begin{Shaded}
\begin{Highlighting}[]
\FunctionTok{ggseasonplot}\NormalTok{(w, }\AttributeTok{polar =} \ConstantTok{FALSE}\NormalTok{, }\AttributeTok{main =} \StringTok{"Wykres sezonowości dla lat 2006{-}2019"}\NormalTok{, }\AttributeTok{xlab =} \ConstantTok{NULL}\NormalTok{) }\SpecialCharTok{+} 
  \FunctionTok{theme}\NormalTok{(}\AttributeTok{legend.position =} \StringTok{"none"}\NormalTok{)}
\end{Highlighting}
\end{Shaded}

\includegraphics{HardCoalDataAnalysis_files/figure-latex/unnamed-chunk-13-2.pdf}

Z wykresu sezonowości widzimy powtarzający się trend wzrostu cen węgla
kamiennego w okresie od sierpnia do listopada. W okresie od stycznia do
maja zauważalny jest nieznaczny trend spadkowy cen.

\end{document}
